\subsection{One MPI rank per node}
\begin{figure}[htpb]
    \centering
    \incfig[1]{gflops\_2x4\_multi\_rome16q}
    \caption{Achieved GFLOPS on multi node setup using the \romeq{} chip.}
    \label{fig:gflopsromemulti}
\end{figure}

Figure \ref{fig:gflopsromemulti} illustrates the achieved GFLOPS on up to 8 single socketed nodes using the \romeq{} chip. Notice how strategy A exhibits a "zig-zag" pattern depening on whether or not the number of nodes is even or odd. This is due to a quirk of how MPI decides to work (see \cite{10064025}).

\begin{figure}[htpb]
    \centering
    \incfig[1]{t\_2x4\_multi\_rome16q}
    \caption{}
    \label{fig:tromemulti}
\end{figure}

\begin{figure}[htpb]
    \centering
    \incfig[1]{tcomm\_2x4\_multi\_rome16q}
    \caption{}
    \label{fig:tcommromemulti}
\end{figure}

\begin{figure}[htpb]
    \centering
    \incfig{tcomp\_2x4\_multi\_rome16q}
    \caption{Tcomp 2X4 Multi Rome16Q}
    \label{fig:tcomp_2x4_multi_rome16q}
\end{figure}


\begin{figure}[htpb]
    \centering
    \incfig[1]{commload\_2x4\_multi\_rome16q}
    \caption{}
    \label{fig:commlaoadromemulti}
\end{figure}
