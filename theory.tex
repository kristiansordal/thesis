
\chapter{Theory}

\section{Definitions}

\begin{definition}[Separator]
    In the context of SpMV, a separator is a node in the graph that has an edge that strides between two partitions.
\end{definition}

\section{Amdahl's Law}

Amdahl’s Law provides a theoretical framework for understanding the limits of performance improvement when additional computational resources are applied to a given problem. It quantifies the potential speedup achieved by optimizing a specific portion of a system, emphasizing that the overall gain is constrained by the proportion of time the optimized component contributes to execution.

\begin{definition}[Amdahl’s Law] The maximum achievable speedup of a computation is limited by the fraction of execution time that remains sequential, even when an arbitrarily large number of parallel resources is employed. \end{definition}

In the context of parallel computing, this principle highlights that while increasing the number of processing units can accelerate the parallelizable portion of a workload, the sequential fraction imposes a fundamental performance ceiling. Formally, if \(S\)  denotes the total speedup, \(t_{p}\) represents the fraction of execution time that can be parallelized, and \(s_{p}\) is the speedup achieved for that parallelizable portion, Amdahl’s Law is expressed as:
\begin{equation} S = \frac{1}{(1 - t_p) + \frac{t_p}{s_p}} \end{equation}
This equation reveals that as \(s_{p} \rightarrow \infty\), the theoretical maximum speedup approaches \(\frac{1}{1-t_{p}}\), illustrating that the non-parallelizable portion becomes the dominant limiting factor in scalability.



