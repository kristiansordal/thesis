
\chapter{Theory}

\section{Sparse Matrix-Vector Multiplication}
Sparse Matrix-Vector Multiplication (SpMV) is a fundamental operation encountered in many areas of scientific computing. It is especially prominent in solving large systems of linear equations and in large-scale simulations. The matrices involved are typically both very large and very sparse. 

A matrix can in theory be considered sparse if it is worthwhile to treat zero values separately. In theory, this translates to a matrix being less than full, i.e. less than \( \mathcal{O}\left(n^2\right)\) nonzeros for a \(n \times  n\) matrix. However, in the context of sparse linear algebra, sparse means that there is a constant number of nonzeros per row, i.e. \(\mathcal{O}\left(n\right)\) nonzeros per row. The matrices used in scientific computing, such as matrices based on meshes, or graphs such as social networks all have this property. Optimizing the performance of SpMV, particularly through parallel computing techiniques, is crucial for enhancing the efficiency of many scientific applications.

However, SpMV is notoriously difficult to optimize, both in sequential and parallel implementations. One major reason is its inherently low computational intensity.

% \section{Definitions}

% \begin{definition}[Separator]
%     In the context of SpMV, a separator is a node in the graph that has an edge that strides between two partitions.
% \end{definition}

% \section{Amdahl's Law}

% Amdahl’s Law provides a theoretical framework for understanding the limits of performance improvement when additional computational resources are applied to a given problem. It quantifies the potential speedup achieved by optimizing a specific portion of a system, emphasizing that the overall gain is constrained by the proportion of time the optimized component contributes to execution.

% \begin{definition}[Amdahl’s Law] The maximum achievable speedup of a computation is limited by the fraction of execution time that remains sequential, even when an arbitrarily large number of parallel resources is employed. \end{definition}

% In the context of parallel computing, this principle highlights that while increasing the number of processing units can accelerate the parallelizable portion of a workload, the sequential fraction imposes a fundamental performance ceiling. Formally, if \(S\)  denotes the total speedup, \(t_{p}\) represents the fraction of execution time that can be parallelized, and \(s_{p}\) is the speedup achieved for that parallelizable portion, Amdahl’s Law is expressed as:
% \begin{equation} S = \frac{1}{(1 - t_p) + \frac{t_p}{s_p}} \end{equation}
% This equation reveals that as \(s_{p} \rightarrow \infty\), the theoretical maximum speedup approaches \(\frac{1}{1-t_{p}}\), illustrating that the non-parallelizable portion becomes the dominant limiting factor in scalability.

\section{Definitions}

\begin{table}[H]
    \begin{center}
        \begin{tabular}[c]{|p{4cm}|p{8cm}|}
            \hline
             \textbf{Term}&\textbf{Definition}  \\
            \hline
             Node&A node, or a compute node, is a computeunit within a larger parallel computing system.    \\
            \hline
                 Dual/single socket&A dual or single socket node refers to the amount of processors on a node. Dual socketed nodes have two processors, that are connected through an interconnect.\\
            \hline
        \end{tabular}
    \end{center}
\end{table}

\section{Latency}
The execution time of computational operations can vary significantly, and it is important to have some understanding of the latency times associated with typical operations. Referencing these latency numbers can be valuable when interpreting benchmark results and the performance of different programs.
\begin{table}[H]
    \begin{center}
        \begin{tabular}[|l|l|l|]{}
            \hline
             L1 cache reference&&  \\
             Branch misprediction&&  \\
             L2 cache reference&&  \\
               Mutex lock/unlock &&  \\
             Branch misprediction&&  \\
             Branch misprediction&&  \\
             Branch misprediction&&  \\
             Branch misprediction&&  \\
             Branch misprediction&&  \\
             Branch misprediction&&  \\
             Branch misprediction&&  \\
             Branch misprediction&&  \\
            \hline
        \end{tabular}
    \end{center}
\end{table}


\section{Parallel Architectures}
There are two main architectures used in the parallel computing industry: Shared Memory Architecture and Distributed Memory Architecture. The following sections gives an overview of the key difference between the two.

\subsection{Shared Memory Architecture}
On a system with shared memory architecture, every processing unit (PU) have access to the same memory, treat it as a global address space. On such systems, the biggest challenge is that of \textit{cache coherency}, where in order to prevent race conditions, every read of the cache must reflect the latest write (adapted from \cite{manchanda2010non}). 

\subsection{Distributed Memory Architecture}
On systems with distributed memory architecture, every processor have their own local memory, not accesible by other processors. When a process needs to access memory from another process, explicit communication of the data stored at that memory address needs to occur, and happens through whichever network the processors are connected with (adapted from \cite{manchanda2010non}). Figure \ref{fig:sharedmemory} shows the difference between shared and distributed memory architectures.

\begin{figure}[ht]
    \centering
    \incfig{sharedmemory}
    \caption{Shared and Distributed Memory Architecture (adapted from \cite{llnlparallel})}
    \label{fig:sharedmemory}
\end{figure}

\subsection{Non-Uniform Memory Access}
Non-Uniform Memory Access (NUMA) refers to a multiprocessor system architecture in which memory access latencies depend on the location of the memory relative to a given processor. Unlike traditional symmetric multiprocessing (SMP) systems, where all processors share equal access times to a centralized memory pool, NUMA architectures consist of multiple processor sockets or nodes, each directly connected to its own local memory. These nodes are interconnected by a high-speed communication network, typically an interconnect, facilitating access to remote memory residing on other nodes.

\subsection{IPUs and emerging architectures}
In the ever evolving world of computer chip manufacturing, there has in the past years emerged a new trend in the field of architecture design. This comes in the form of packaging thousands of small processor cores into a single device, where each core has its own local memory, and no device-level shared memory \cite{10.3389/fphy.2023.979699}. An example of such a processor is the relatively new Cerebras Wafer-Scale Engine (WSE-3), which is the largest chip ever built. With a spec sheet of 4 trillion transistors, 900 000 cores, memory bandwidth of 21PB/s, and 44 GB of on-wafer memory (see \cite{cerebras2024wse3}). Such architectures necessitates a careful distribution of relevant data across each processor, and is where memory scalable communication strategies are beneficial.
\medskip
